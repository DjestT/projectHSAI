\documentclass[12pt]{article}

\author{Dylan Toirkens, Joey Wilmots}
\title{\textbf{TitelBestand}}
\date{11/10/2017}

\usepackage{graphicx}
\usepackage{float}
\usepackage[dutch]{babel}
\usepackage{enumerate}


\setcounter{tocdepth}{3}
\setcounter{secnumdepth}{3}
\begin{document}
	\begin{titlepage}
		
		\newcommand{\HRule}{\rule{\linewidth}{0.5mm}} % Defines a new command for the horizontal lines, change thickness here
		
		\begin{center} % Center everything on the page
			
			\textsc{\LARGE Universiteit Hasselt}\\[1.5cm] % Nme of your university/college
			\textsc{\Large Humane en sociale aspecten van de informatica}\\[0.5cm] % Major heading such as course name
			
			\HRule \\[0.4cm]
			{ \huge \bfseries Project: Aannames, scenario, taakanalyse en planning}\\[0.4cm]
			\HRule \\[1.5cm]
			
			\begin{minipage}{0.5\textwidth}
				\begin{flushleft} \large
					Groep 10:\newline
					Dylan \textsc{Toirkens}\newline
					Joey \textsc{Wilmots}
				\end{flushleft}
			\end{minipage}
			~
			\begin{minipage}{0.3\textwidth}
				\begin{flushright} \large
					\emph{Datum:}\\
					11 november 2017
					\emph{Academiejaar: } \\
					2017-2018
				\end{flushright}
			\end{minipage}\\[3cm]
			\vspace{25 mm}
			\includegraphics[width=3.0cm]{UHasselt-logo.jpg}\\[2.0cm]  
		\end{center}
	\end{titlepage}
\newpage
\tableofcontents
\clearpage

\section{Aannames}

\vspace{5mm}

\begin{itemize}
	\item De tablet beschikt over voldoende schermgrootte (ongeveer 10 inch), zodat alle afbeeldingen en toetsen in duidelijke grootte afgebeeld worden voor kleuters.
	\item De gebruikers hebben zo goed als geen lees- en schrijfvaardigheden waardoor de GUI zal bestaan uit intu\"itieve toetsen en duidelijke afbeeldingen.
	\item De leerlingen hebben enkele inleidende lessen gekregen die ze wegwijs maken met de tablet.
	\item De tablet zal ook beschikken over een trilmotor die feedback geeft op schermaanrakingen.
	\item De tablet is uitgerust met een manier om het scherm te projecteren (bijvoorbeeld m.b.v. chromecast).
	\item Auditieve feedback is aanwezig om uitleg te geven over de dieren.
	\item Er is een AUX-input en een koptelefoon voorzien bij elk apparaat. 
	\item Het programma kan in lesverband gebruikt worden maar ook als ontspanning voor de leerlingen, die daarbij ook educatief is.
	\item Het programma is ideaal om als voorkennis te gebruiken indien een trip naar de Zoo gepland staat.
\end{itemize}

\clearpage

\section{Scenario}

\vspace{5mm}

Na het verloop van de les kunnen de leerlingen gebruik maken van een tablet met enkele apps ter ontspanning. Hieronder een dierenapp met een groot startscherm met foto's van dieren. Er is een volgende en vorige knop die een nieuwe pagina van foto's toont. De foto's zijn klikbaar waarna er een animatie afspeelt met bijkomende uitleg over het dier. Zo is de uitleg zowel visueel als auditief.\\
Na afloop van de animatie kunnen de kinderen aan de hand van een spelvorm hun kennis testen door het dier het juiste voer te geven, of ermee te spelen/aaien terwijl er allerlei weetjes verteld worden.\\
De animatie kan ook opnieuw afgespeeld worden, waarvoor het reset logo gebruikt wordt aangezien deze een intu\"itieve feel geeft. Aan de hand van een vorige knop, of de hardware home knop kan men terug navigeren naar de startpagina waar de leerling uit een reeks foto's van dieren kan kiezen.\\ 
In lesverband kan de leerkracht het scherm projecteren zodat de gehele klas mee kan kijken. Animaties kunnen gepauzeerd worden waarbij de leerkracht dus extra uitleg kan geven, of vragen gesteld kunnen worden. Na afloop kan de leerkracht gebruik maken van een ondervragingssysteem met meerkeuze vragen. Deze meerkeuze test kan geactiveerd en gedeactiveerd worden in de settings van de app (de settings zal in een kleiner icoon gedisplayed worden aangezien het niet de bedoeling is dat de leerlingen hiermee gaan spelen). De leerkracht leest de vraag en de mogelijkheden voor waarna de leerlingen kunnen antwoorden. De leerkracht duid het gegeven antwoord aan en indien er een fout antwoord gegeven wordt zal het geanimeerde dier vertellen wat het juiste antwoord is en waarom, alsook waarom het gegeven antwoord incorrect is. In geval van een goed antwoord zal positieve feedback gegeven worden.

\clearpage


\section{Taakanalyse}
\begin{enumerate}
	\item Start de applicatie
	\item Selecteer een mogelijkheid
	\begin{enumerate}[(1)]
		\item Foto's bekijken
		\begin{enumerate}[1.]
			\item Klik op de foto of het dier om uitleg te krijgen en een animatie te starten
			\item Klik op de pijl naar rechts voor de volgende foto
			\item Klik op de pijl naar links voor de vorige foto
			\item Klik op de home knop om terug te gaan naar het menu
		\end{enumerate}
		\item Spel spelen en kies een spel
		\begin{enumerate}[1.]
			\item Dieren aaien en voedens
			\begin{enumerate}[(1)]
				\item Klik op het dier
				\item Klik op een knop om het dier te aaien
				\item Kies een voedingsproduct om het dier te voeden
				\item Klik op het kruisje om een ander dier te kiezen
				\item Klik op de home knop om terug te gaan naar het menu
			\end{enumerate}
			\item Dierengeluiden raden
			\begin{enumerate}[(1)]
				\item Luister naar het dierengeluid
				\item Klik op de microfoon knop om het geluid opnieuw af te spelen
				\item Klik op een dier
				\item Klik op de home knop om terug te gaan naar het menu
			\end{enumerate}
			\item Dierenquiz
			\begin{enumerate}[(1)]
				\item Luiser naar de vraag
				\item Klik op de microfoon knop om de vraag opnieuw af te spelen
				\item klik op een dier
				\item Klik op de home knop om terug te gaan naar het menu
			\end{enumerate}
		\end{enumerate}
		\item Geluidsinstellingen
		\begin{enumerate}[1.]
			\item Pas het geluid aan
			\item Sla de instellingen op
			\item Klik op de home knop om terug te gaan naar het menu
		\end{enumerate}
	\end{enumerate}
\end{enumerate}

\textbf{Plan 0} Doe stappen 1 en 2 in volgorde.

\textbf{Plan 2} Doe, indien gewenst, stappen 2.1, 2.2 of 2.3.

\textbf{Plan 2.1} Doe stap 2.1.1. Doe vervolgens, indien gewenst, stappen 2.1.2 en 2.1.3. Doe stap 2.1.4. Stappen 2.1.1 tot en met 2.1.4 kunnen zo vaak als gewenst herhaald worden. Op elk moment kan stap 2.1.5 gedaan worden.

\textbf{Plan 2.2} Doe stap 2.2.1. Doe vervolgens, indien gewenst, stap 2.2.2. Doe stap 2.2.3. Stappen 2.2.1 tot en met 2.2.3 kunnen zo vaak als gewenst herhaald worden. Op elk moment kan stap 2.2.4 gedaan worden.

\textbf{Plan 2.3} Doe indien gewenst, stap 2.3.1 en vervolgens 2.3.2. Eindig met stap 2.3.3.
\section {Planning en taakverdeling}
\begin{itemize}
	\item 11/10: Analyse schrijven
	\begin{itemize}
		\item Aannames: Joey
		\item Scenario: Joey
		\item Taakanalyse: Dylan
		\item Planning: Dylan
	\end{itemize}
	\item 18/10: Paper mockups maken: samen
	\item 25/10: Paper mockups maken: samen
	\item 08/11: Paper mockups afwerken en presentatie voorbereiden: samen
	\item 12/11: Inleveren paper mockups
	\item 15/11: Presentatie mockups
	\item 29/11: Prototype 1 maken: samen
	\item 06/12: Prototype 1 maken: samen
	\item 13/12: Prototype 1 afwerken: samen
	\item 17/12: Inleveren prototype 1
	\item 20/12: Reviews prototype van andere groepen
	\item 07/01: Inleveren prototype 2 en rapport
\end{itemize}
\end{document}

